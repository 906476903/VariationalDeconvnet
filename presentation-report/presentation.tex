%%%%%%%%%%%%%%%%%%%%%%%%%%%%%%%%%%%%%%%%%
% Beamer Presentation
% LaTeX Template
% Version 1.0 (10/11/12)
%
% This template has been downloaded from:
% http://www.LaTeXTemplates.com
%
% License:
% CC BY-NC-SA 3.0 (http://creativecommons.org/licenses/by-nc-sa/3.0/)
%
%%%%%%%%%%%%%%%%%%%%%%%%%%%%%%%%%%%%%%%%%

%----------------------------------------------------------------------------------------
%	PACKAGES AND THEMES
%----------------------------------------------------------------------------------------

\documentclass{beamer}

\mode<presentation> {

% The Beamer class comes with a number of default slide themes
% which change the colors and layouts of slides. Below this is a list
% of all the themes, uncomment each in turn to see what they look like.

%\usetheme{default}
%\usetheme{AnnArbor}
%\usetheme{Antibes}
%\usetheme{Bergen}
%\usetheme{Berkeley}
%\usetheme{Berlin}
%\usetheme{Boadilla}
%\usetheme{CambridgeUS}
%\usetheme{Copenhagen}
%\usetheme{Darmstadt}
%\usetheme{Dresden}
%\usetheme{Frankfurt}
%\usetheme{Goettingen}
%\usetheme{Hannover}
%\usetheme{Ilmenau}
%\usetheme{JuanLesPins}
%\usetheme{Luebeck}
%\usetheme{Madrid}
%\usetheme{Malmoe}
%\usetheme{Marburg}
%\usetheme{Montpellier}
%\usetheme{PaloAlto}
%\usetheme{Pittsburgh}
\usetheme{Rochester}
%\usetheme{Singapore}
%\usetheme{Szeged}
%\usetheme{Warsaw}

% As well as themes, the Beamer class has a number of color themes
% for any slide theme. Uncomment each of these in turn to see how it
% changes the colors of your current slide theme.

%\usecolortheme{albatross}
%\usecolortheme{beaver}
%\usecolortheme{beetle}
%\usecolortheme{crane}
%\usecolortheme{dolphin}
%\usecolortheme{dove}
%\usecolortheme{fly}
\usecolortheme{lily}
%\usecolortheme{orchid}
%\usecolortheme{rose}
%\usecolortheme{seagull}
%\usecolortheme{seahorse}
%\usecolortheme{whale}
%\usecolortheme{wolverine}

% \setbeamertemplate{footline} % To remove the footer line in all slides uncomment this line
\setbeamertemplate{footline}[page number] % To replace the footer line in all slides with a simple slide count uncomment this line

\setbeamertemplate{navigation symbols}{} % To remove the navigation symbols from the bottom of all slides uncomment this line
\setbeamertemplate{bibliography item}{} %Remove icons in bibliography
}

\usepackage{graphicx} % Allows including images
\usepackage{booktabs} % Allows the use of \toprule, \midrule and \bottomrule in tables
\usepackage{amsmath}
\usepackage{lmodern}

%----------------------------------------------------------------------------------------
%	TITLE PAGE
%----------------------------------------------------------------------------------------

\title[Project Learning Systems]{Unsupervised training of Convolutional Networks with SGVB} % The short title appears at the bottom of every slide, the full title is only on the title page

\author{Joost van Amersfoort \& Otto Fabius} % Your name
\institute[UvA] % Your institution as it will appear on the bottom of every slide, may be shorthand to save space
{University of Amsterdam
% Your institution for the title page
\medskip
}
\date{\today} % Date, can be changed to a custom date

\begin{document}

\begin{frame}
\titlepage % Print the title page as the first slide
\end{frame}

\begin{frame}
\frametitle{Overview} % Table of contents slide, comment this block out to remove it
\tableofcontents % Throughout your presentation, if you choose to use \section{} and \subsection{} commands, these will automatically be printed on this slide as an overview of your presentation
\end{frame}

%----------------------------------------------------------------------------------------
%	PRESENTATION SLIDES
%----------------------------------------------------------------------------------------

%------------------------------------------------
\section{Introduction} % Sections can be created in order to organize your presentation into discrete blocks, all sections and subsections are automatically printed in the table of contents as an overview of the talk
%------------------------------------------------

\begin{frame}
\frametitle{Introduction}
\begin{itemize}
	\item Convolutional Networks have had much success recently in Computer Vision tasks, due efficient implementation and an increase in computing power and available annotated data.
	\item However, to date, no successful means of training such Networks unsupervised has been reported.
	\item The recently developed SGVB makes efficient, effective Bayesian Inference possible by means of stochastic gradient descent. This can be used to train a ConvNet unsupervised
\end{itemize}
\end{frame}

\section{Theory}

\subsection{Convolutional Neural Networks}

\begin{frame}
\frametitle{Convolutional Neural Networks - Error Backpropagation}
The gradient of a convolution is relatively straightforward.\\ Let $L$ be the loss function, and $y = x$ \textasteriskcentered $k$ the convolution operation with weights $W$, then \\   
\begin{align*}
\nabla_W L = (\nabla_y L) \ast x^T 
\end{align*}
\end{frame}

\begin{frame}
\frametitle{Convolutional Neural Networks - Typical structure}

%image of example structure
%tell something about need for computational resources
\end{frame}

\subsection{SGVB}
\begin{frame}
\frametitle{Unsupervised Training with SGVB}
%image of example structure with latent variables and deconvolution
Model $P(\mathbf{X},\mathbf{Z}) = \frac{P(\mathbf{Z})\cdot P(\mathbf{X}|\mathbf{Z})}{P(\mathbf{Z}|\mathbf{X})}$ to find structure in data $\mathbf{X}$. \vspace{0.5mm}
Here, $P(\mathbf{X}|\mathbf{Z})$ is the deconvolutional layer and the intractable $P(\mathbf{Z}|\mathbf{X})$ is approximated by $q(\mathbf{Z}|\mathbf{X})$, which is the Convolutional layer.
%mention that we need to train the parameters of the network and that we can use this for pretraining an convnet, but also have a generative model which can be used for many different tasks.
\end{frame}

\begin{frame}
\frametitle{Marginal Likelihood as Optimization Criterion}
%Of course, we want to learn the parameters via gradient descent on some objective function
The marginal probability of $\mathbf{X}$ can be written as:
\begin{align*}
\log p_\theta(\mathbf{X}) = D_{KL}(q_\phi(\mathbf{Z}|\mathbf{X}) || p_\theta(\mathbf{Z}|\mathbf{X})) + \mathcal{L}(\mathbf{\theta}, \mathbf{\phi}; \mathbf{X})
\end{align*}
Where 
$\mathcal{L}(\mathbf{\theta}, \mathbf{\phi}; \mathbf{X})$
is the \textit{lower bound} on the marginal likelihood
$
P(\mathbf{X}) = \int_z P(\mathbf{Z})P(\mathbf{X}|\mathbf{Z})d\mathbf{Z}$
\end{frame}

\begin{frame}
\frametitle{Reparameterization}
We need to estimate (stochastic) gradients w.r.t the model parameters. For this, we reparameterize $q(\mathbf{Z}|\mathbf{X})$ as a differentiable, deterministic function $g_{\theta}(\epsilon,\mathbf{x})$, which depends on sampled noise $\epsilon \sim N(0,1)$. \\ Now we can sample $\tilde{z} \sim q(\mathbf{Z}|\mathbf{X})$ and we can perform stochastic gradient descent on the model parameters!
\end{frame}

\section{Research Questions}

\section{Implementation}

\subsection{Torch7}

\subsection{GPU}
 
\section{Results}

\section{Discussion}


%-----------------------------------------------
%\begin{frame}[shrink=20]{Bibliography}
%	\bibliographystyle{ieeetr}
%	\bibliography{ref}
%\end{frame}

%------------------------------------------------

\begin{frame}
\Huge{\centerline{Thank you for your attention!}}
\end{frame}

%----------------------------------------------------------------------------------------

\end{document} 
